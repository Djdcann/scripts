\documentclass{article}

\usepackage[top=2.54cm, left=2.54cm, right=2.54cm, bottom=2.54cm]{geometry}
\usepackage{amsmath}
\usepackage{graphicx}
\usepackage{booktabs}
\usepackage{hyperref}
\usepackage{multicol}
\hypersetup{colorlinks=true, urlcolor=blue,}
\usepackage[svgnames]{xcolor}

\begin{document}

\hrule
\begin{center}
\large {Quantum Harmonic Oscillator}\\ \large{and Hermite Polynomials}
\end{center}

\hrule
\vspace{1pt}
\hrule height 1pt

\section{Introduction}

\noindent The Schrodinger equation for a harmonic oscillator may be obtained by using the classical spring potential given by

\begin{equation}
\label{eq:1}
V(x)=\frac{1}{2}m\omega^{2}x^{2},
\end{equation}
where $m$ is the mass, $x$ is the position, and $\omega$ is the angular frequency given by

\begin{equation}
\label{eq:2}
\omega=\sqrt{\frac{k}{m}}
\end{equation}
The Schrodinger equation with this form of potential is described by

\begin{equation}
\label{eq:3}
-\frac{\hbar^{2}}{2m}\frac{d^{2}\Psi(x)}{dx^{2}}+V(x)\Psi(x)=E\Psi(x)
\end{equation}
where $V(x)$ is given by Eq.~\eqref{eq:1} Since the derivative of the wavefunction must give back the square of $x$ plus a constant times the original function, then the form

\begin{equation}
\label{eq:4}
\Psi(x)=Ce^{-\alpha x^{2}/2}
\end{equation}
is suggested.\\

\noindent Note that this form which is in a Gaussian form satisfies the requirement of going to zero at infinity, making it possible to normalize the wavefunction. Substituting this function into the Schrodinger equation and fitting the boundary conditions leads to the ground state energy for the quantum harmonic oscillator given by

\begin{equation}
\label{eq:5}
E_0=\frac{\hbar\omega}{2}
\end{equation}
The general solution to the Schrodinger equation leads to a sequence of evenly spaced energy levels characterized by a quantum number $n$ as shown in Fig.~\ref{fig:1}.

\begin{figure}[h]
\begin{center}
\includegraphics[scale=0.65]{potwell.eps}
\caption{\label{fig:1}Image Credit: \url{www.hyperphysics.com}}
\end{center}
\end{figure}

\noindent The wavefunctions for the quantum harmonic oscillator contain the Gaussian form which allows them to satisfy the necessary boundary conditions at infinity. In the wavefunction associated with a given value of the quantum number $n$, the Gaussian is multiplied by a polynomial of order $n$ called a \textbf{Hermite polynomial}. The expressions are simplified by making the substitution

\begin{equation}
\label{eq:6}
y=\sqrt{\alpha}x
\end{equation}
where $\alpha=\dfrac{m\omega}{\hbar}$. The general formula for the normalized wavefunction is then

\begin{equation}
\label{eq:7}
\Psi(x)=(y)=\Big(\frac{\alpha}{\pi}\Big)^{1/4}\frac{1}{\sqrt{2^{n}n!}}H_n(y)e^{-y^{2}/2}
\end{equation}
where $H_n(y)$ are the Hermite Polynomials.

\section{Solutions to the Schrodinger Equation}

\noindent The Schrodinger equation for a harmonic oscillator may be solved to give the wavefunctions illustrated in Fig.~\ref{fig:2}

\begin{figure}[h]
\begin{center}
\includegraphics{potwell.eps}
\caption{\label{fig:2}Solutions to the Schrodinger Equation for the first six energy states gives the normalized wavefunctions shown here.}
\end{center}
\end{figure}

\noindent The probability of finding the oscillator at any given value of $x$ is the square of the wavefunction. Note that the wavefunctions for higher $n$ have more “humps” within the potential well. This corresponds to a shorter wavelength and therefore by the deBroglie relationship they may be seen to have a higher momentum and therefore higher energy.\\

\noindent The most probable value of position for the lower states is very different from the classical harmonic oscillator where it spends more time near the end of its motion. But as the quantum number increases, the probability distribution becomes more like that of the classical oscillator.\\

\noindent When the Schrodinger equation for the harmonic oscillator is solved by a series method, the solutions contain this set of polynomials, named the Hermite polynomials. The values for $n$ for the first six are shown in Table~\ref{tab:1}.

\begin{table}[ht]
\begin{center}
\caption{\label{tab:1} blah}
\begin{tabular}{ l l r }
  \hline
  \hline
  $n$ & $H_n(y)$ & $E_n$ \\
  \hline
  0 & $1$ & $\frac{1}{2}\hbar\omega$ \\[0.5em]
  1 & $2y$ & $\frac{3}{2}\hbar\omega$ \\[0.5em]
  2 & $4y^{2}-2$ & $\frac{5}{2}\hbar\omega$ \\[0.5em]
  3 & $8y^{3}-12y$ & $\frac{7}{2}\hbar\omega$ \\[0.5em]
  4 & $16y^{4}-48y^{2}+12$ & $\frac{9}{2}\hbar\omega$ \\[0.5em]
  5 & $32y^{5}-160y^{3}+120y$ & $\frac{11}{2}\hbar\omega$ \\[0.5em]
  \hline
  \hline
\end{tabular}
\end{center}
\end{table}

\noindent The wavefunctions for the quantum harmonic oscillator contain the Gaussian form which allows them to satisfy the necessary boundary conditions at infinity. In the wavefunction associated with a given value of the quantum number $n$, the Gaussian is multiplied by a polynomial of order $n$ (the Hermite polynomials
above) and the constants necessary to normalize the wavefunctions.

\end{document}
